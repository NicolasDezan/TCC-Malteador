% RESUMO - PT
\begin{resumo}
  \vspace{-15pt}
  
  A viabilização de pesquisas em malteação demanda equipamentos capazes de controlar variáveis como temperatura, umidade e tempo. Contudo, soluções comerciais nem sempre são acessíveis para uso acadêmico, especialmente em laboratórios que operam com recursos limitados. Este trabalho apresenta o desenvolvimento de um sistema de automação baseado em um microcontrolador \textit{ESP32-C3}, escolhido principalmente pela sua relação custo-benefício, com \textit{firmware} programado em \textit{MicroPython} e um aplicativo desenvolvido em \textit{Kotlin}, a linguagem oficial da plataforma \textit{Android}, capaz de monitorar e controlar o processo de malteação em laboratório. O sistema permite a configuração remota dos parâmetros de processo e o monitoramento em tempo real por meio da comunicação via \textit{Bluetooth Low Energy} (BLE). A arquitetura do \textit{firmware} foi montada com estrutura assíncrona, permitindo que múltiplos processos, como leitura de sensores e comunicação \textit{Bluetooth}, ocorram simultaneamente sem bloqueios. O aplicativo \textit{Android} adota uma arquitetura moderna em camadas, que separa as responsabilidades entre acesso a dados, lógica de negócio e interface do usuário com interface construída em \textit{Jetpack Compose}. Os códigos desenvolvidos estão disponibilizados em repositórios \textit{GitHub}, com o objetivo de promover reprodutibilidade e facilitar a adoção por outros pesquisadores. A solução proposta visa atender às demandas do LACEMP-IFES por uma plataforma de baixo custo e alta confiabilidade, contribuindo para a pesquisa aplicada em tecnologia de alimentos.

  Palavras-chave: \palavraschaveemlinha
\end{resumo}


% RESUMO - EN
\begin{resumo}[Abstract]
  \vspace{-15pt}
  
  \begin{otherlanguage*}{english}
  The feasibility of malting research requires equipment capable of controlling variables such as temperature, humidity, and time. However, commercial solutions are not always accessible for academic use, especially in laboratories with limited resources. This work presents the development of an automation system based on the ESP32-C3 microcontroller, selected primarily for its cost-effectiveness. The firmware was developed in MicroPython, while a companion Android application was built in Kotlin, the platform’s official language. The system enables remote configuration of process parameters and real-time monitoring via Bluetooth Low Energy (BLE) communication. The firmware architecture is designed using an asynchronous structure, allowing multiple processes—such as sensor reading and Bluetooth communication—to run concurrently without blocking. The Android app follows a modern layered architecture that separates concerns across data access, business logic, and user interface, which was developed using Jetpack Compose. All source code is available on GitHub to promote reproducibility and support adoption by other researchers. This solution aims to meet the needs of LACEMP-IFES by providing a low-cost, reliable platform that contributes to applied research in food technology.

  
  Keywords: \inlinekeywords
\end{otherlanguage*}
\end{resumo}