% RESUMO - PT
\begin{resumo}
  \vspace{-15pt}
  
  A malteação é uma etapa crítica na produção de cerveja, exigindo o controle rigoroso de variáveis como temperatura, umidade e tempo ao longo das fases de maceração, germinação e secagem. Em ambientes acadêmicos, a ausência de equipamentos acessíveis e confiáveis limita o desenvolvimento de pesquisas nesta área. Este trabalho apresenta o desenvolvimento de um sistema de automação baseado no microcontrolador ESP32, com firmware programado em MicroPython e um aplicativo Android desenvolvido em Kotlin, capaz de monitorar e controlar o processo de malteação em laboratório. O sistema permite a configuração remota dos parâmetros de processo e o monitoramento em tempo real por meio da comunicação via Bluetooth Low Energy (BLE). A arquitetura do firmware foi estruturada com tarefas assíncronas independentes para controle de sensores, atuadores e etapas do processo. O aplicativo Android adota a arquitetura moderna em camadas (data, domain, presentation) com interface responsiva em Jetpack Compose, proporcionando usabilidade e modularidade. Os códigos desenvolvidos estão disponibilizados em repositórios GitHub, com o objetivo de promover reprodutibilidade e facilitar a adoção por outros pesquisadores. A solução proposta visa atender às demandas do LACEMP-IFES por uma plataforma de baixo custo e alta eficiência, contribuindo para a pesquisa aplicada em tecnologia de alimentos e automação de processos.

  Palavras-chave: \palavraschaveemlinha
\end{resumo}


% RESUMO - EN
\begin{resumo}[Abstract]
  \vspace{-15pt}
  
  \begin{otherlanguage*}{english}
    Malting is a critical stage in beer production, requiring strict control of variables such as temperature, humidity, and time during the steeping, germination, and kilning phases. In academic environments, the lack of accessible and reliable equipment limits research progress in this field. This work presents the development of an automation system based on the ESP32 microcontroller, with firmware programmed in MicroPython and an Android application developed in Kotlin, capable of monitoring and controlling the malting process in a laboratory setting. The system enables remote configuration of process parameters and real-time monitoring via Bluetooth Low Energy (BLE) communication. The firmware architecture was structured with independent asynchronous tasks to handle sensors, actuators, and process stages. The Android application adopts a modern layered architecture (data, domain, presentation) with a responsive interface built using Jetpack Compose, providing usability and modularity. The developed code is publicly available on GitHub, aiming to promote reproducibility and facilitate adoption by other researchers. The proposed solution addresses the demands of LACEMP-IFES for a low-cost and efficient platform, contributing to applied research in food technology and process automation.
  
  Keywords: \inlinekeywords
\end{otherlanguage*}
\end{resumo}