% RESUMO - PT
\begin{resumo}
  \vspace{-15pt}
  
  A viabilização de pesquisas em malteação demanda equipamentos capazes de controlar variáveis como temperatura, umidade e tempo. Contudo, soluções comerciais nem sempre são acessíveis para uso acadêmico, especialmente em laboratórios que operam com recursos limitados. Este trabalho apresenta o desenvolvimento de um sistema de automação baseado em um microcontrolador /textit{ESP32-C3}, escolhido principalmente pela sua relação custo-benefício, com /textit{firmware} programado em /textit{MicroPython} e um aplicativo /textit{Android} desenvolvido em /textit{Kotlin}, a linguagem oficial da plataforma /textit{Android}, capaz de monitorar e controlar o processo de malteação em laboratório. O sistema permite a configuração remota dos parâmetros de processo e o monitoramento em tempo real por meio da comunicação via /textit{Bluetooth Low Energy} (BLE). A arquitetura do /textit{firmware} foi montada com estrutura assíncrona, permitindo que múltiplos processos, como leitura de sensores e comunicação /textit{Bluetooth}, ocorram simultaneamente sem bloqueios. O aplicativo /textit{Android} adota uma arquitetura moderna em camadas, que separa as responsabilidades entre acesso a dados, lógica de negócio e interface do usuário com interface construída em /textit{Jetpack Compose}, a tecnologia oficial do Android para desenvolvimento de interfaces. Os códigos desenvolvidos estão disponibilizados em repositórios /textit{GitHub}, com o objetivo de promover reprodutibilidade e facilitar a adoção por outros pesquisadores. A solução proposta visa atender às demandas do LACEMP-IFES por uma plataforma de baixo custo e alta eficiência, contribuindo para a pesquisa aplicada em tecnologia de alimentos e automação de processos.

  Palavras-chave: \palavraschaveemlinha
\end{resumo}


% RESUMO - EN
\begin{resumo}[Abstract]
  \vspace{-15pt}
  
  \begin{otherlanguage*}{english}
    The development of research in malting requires equipment capable of controlling variables such as temperature, humidity, and time. However, commercial solutions are not always accessible for academic use, especially in laboratories operating with limited resources. This work presents the development of an automation system based on the ESP32-C3 microcontroller, with firmware programmed in MicroPython and an Android application developed in Kotlin — the official language of the Android platform — capable of monitoring and controlling the malting process in a laboratory environment. The system allows remote configuration of process parameters and real-time monitoring via Bluetooth Low Energy (BLE) communication. The firmware architecture was designed with an asynchronous structure, enabling multiple processes, such as sensor reading and Bluetooth communication, to run simultaneously without blocking. The Android application adopts a modern layered architecture that separates responsibilities between data access, business logic, and user interface, with the interface developed using Jetpack Compose — the official Android toolkit for building declarative and responsive interfaces. The developed code is available in public repositories on GitHub, aiming to promote reproducibility and facilitate adoption by other researchers. The proposed solution meets the demands of LACEMP-IFES for a low-cost and highly efficient platform, contributing to applied research in food technology and process automation.

  
  Keywords: \inlinekeywords
\end{otherlanguage*}
\end{resumo}