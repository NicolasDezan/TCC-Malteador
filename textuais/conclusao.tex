\chapter[Conclusão]{Conclusão}

Este Trabalho de Conclusão de Curso teve como objetivo principal o desenvolvimento de uma solução de software composta por um firmware e um aplicativo Android para automatizar o processo de malteação em ambiente laboratorial. A proposta surgiu como resposta à demanda do LACEMP-IFES, que busca um equipamento acessível para a realização de experimentos controlados de malteação, especialmente em condições climáticas desfavoráveis.

O sistema foi implementado utilizando o microcontrolador ESP32-C3, programado em MicroPython, com algoritmos estruturados de forma assíncrona, capazes de controlar as etapas de maceração, germinação e secagem. O aplicativo Android, desenvolvido em Kotlin com uso do Jetpack Compose, permite a configuração remota de parâmetros, o monitoramento em tempo real de sensores e atuadores, bem como o armazenamento local de receitas para diferentes perfis de malteação. A comunicação entre os dispositivos foi estabelecida via Bluetooth Low Energy (BLE).

Os resultados alcançados demonstraram que a arquitetura modular adotada é satisfatória, sendo flexível e escalável. As funcionalidades essenciais para o controle do processo foram implementadas com sucesso, e o sistema provou ser capaz de manter a comunicação estável, registrar dados operacionais via terminal e responder a comandos externos de forma confiável. Ainda que o protótipo físico original desenvolvido durante a iniciação científica possua limitações que impedem a integração imediata do controle térmico real, a lógica de controle foi devidamente implementada e encontra-se preparada para operação assim que o hardware for ajustado.

Com isso, o presente trabalho entrega não apenas uma plataforma funcional de automação voltada ao malteador laboratorial, mas também uma base modular e documentada de desenvolvimento para sistemas embarcados com ESP32. O firmware e o aplicativo Android, organizados em repositórios públicos, foram estruturados de forma a facilitar sua adaptação para outras aplicações acadêmicas, especialmente na prototipação de equipamentos laboratoriais de baixo custo, como tituladores automáticos, espectrofotômetros e controladores de processo. Assim, além de atender a uma demanda específica do LACEMP-IFES, este trabalho se propõe a servir como ponto de partida para futuros projetos orientados na interseção entre Química e Internet das Coisas, contribuindo com uma trilha prática e acessível para estudantes que desejem explorar a automação em ambientes educacionais e de pesquisa.

Como trabalhos futuros, recomenda-se a montagem e validação experimental do sistema completo com sensores e atuadores reais, a implementação de um módulo adicional para controle da etapa de torrefação (maltes especiais) e a ampliação da interface do aplicativo com funcionalidades voltadas à exportação de dados via Wi-Fi. Essas melhorias ampliarão o potencial da plataforma para experimentos reprodutíveis, análises mais refinadas e aplicação direta em práticas laboratoriais, fortalecendo sua utilidade tanto em pesquisas quanto no ensino técnico sobre a malteação.

Apesar dessas conquistas, destaca-se que a validação prática do sistema em hardware físico completo ainda não foi realizada, sendo este um passo fundamental para futuras aplicações experimentais no LACEMP. A modularidade da arquitetura permite que novos dispositivos sejam integrados facilmente, garantindo escalabilidade. Com os ajustes necessários no protótipo físico, o sistema está pronto para ser incorporado em experimentos reais de malteação, contribuindo para pesquisas sobre novas variedades de grãos, controle de processos e desenvolvimento de tecnologias acessíveis para a indústria cervejeira.