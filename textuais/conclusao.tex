\chapter[Conclusão]{Conclusão}

Este trabalho apresentou o desenvolvimento de um sistema de automação para controle do processo de malteação em escala laboratorial, com base em uma arquitetura composta por \textit{firmware} para o microcontrolador \textit{ESP32-C3} e um aplicativo \textit{Android}. A proposta surgiu da necessidade identificada no \textit{LACEMP-IFES} por uma solução de baixo custo e com potencial didático, voltada ao estudo controlado da malteação em climas quentes.

O \textit{firmware} foi estruturado com lógica assíncrona, permitindo a execução simultânea de múltiplas tarefas, como leitura de sensores, controle de etapas e comunicação via \textit{Bluetooth Low Energy}. Por sua vez, o aplicativo \textit{Android}, desenvolvido em \textit{Kotlin}, adotou uma arquitetura modular em camadas, com funcionalidades que incluem envio de parâmetros, monitoramento em tempo real e gerenciamento de receitas.

Dentro do escopo estabelecido, os objetivos propostos foram alcançados: a comunicação \textit{BLE} entre os dispositivos foi implementada com sucesso, os algoritmos de controle das etapas de maceração, germinação e secagem foram desenvolvidos e validados por meio de simulações, e todo o código foi documentado e disponibilizado em repositórios públicos.

Embora validada por simulações, a solução não foi testada com o protótipo físico em funcionamento, uma vez que sua montagem completa e integração eletromecânica estavam fora do escopo deste trabalho. Assim, os testes com sensores e atuadores reais permanecem como uma etapa futura necessária para avaliar a performance prática da malteação realizada no equipamento.

Dessa forma, o trabalho cumpriu seu papel, entregando uma base funcional e bem documentada para o controle digital do processo. Apesar da ausência de testes com \textit{hardware} real, os resultados obtidos por simulação demonstram a viabilidade da proposta e oferecem um ponto de partida concreto para desenvolvimentos futuros.