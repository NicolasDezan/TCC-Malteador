\chapter[Referencial Teórico]{Referencial Teórico}

\section{O Processo de Malteação}

A malteação é um dos processos fundamentais na indústria cervejeira, responsável pela conversão do grão cru em um ingrediente essencial para a produção de cerveja. Esse processo ocorre em três etapas principais: maceração, germinação e secagem. Essas etapas são fundamentais para a produção de malte base: um malte que possui uma boa quantidade de enzimas e estoques de amido \cite{BRIGGS2004,CENCI2021}. Ainda, pode-se considerar uma quarta etapa no processo: a torrefação, que ocorre após a secagem e produz um malte especial no lugar de um malte base. 

Maltes especiais são conhecidos por serem formados com bastante influência das reações de Maillard \cite{COGHE2004}, conjunto complexo de reações não enzimáticas que envolvem a condensação de açúcares redutores com aminoácidos, gerando compostos intermediários (como os produtos de Amadori) e, enfim, melanoidinas que são responsáveis pela cor e aroma característicos \cite{nursten2005maillard}. A complexidade desse processo está na manipulação de um organismo vivo, o grão, que exige controle rigoroso de suas condições de crescimento para se tornar um produto viável para comercialização \cite{MALLETT2022}. Além disso, em condições inadequadas de malteação, pode ocorrer contaminação dos grãos por fungos dos gêneros \textit{Aspergillus}, \textit{Penicillium} e \textit{Fusarium}, principalmente \cite{LUARASI2016}.

\begin{figure}[ht]
    \centering
    \caption{Fluxograma incluindo as etapas de malteação.}
    \label{fig:fluxogramamalteacao}
    \includegraphics[width=1.0\textwidth]{fluxogramamalteacao.png}

    {\centering\footnotesize Fonte: Autoria própria.\par}
\end{figure}

\subsection{Maceração}

Durante a maceração, com a submersão do grão em água, ocorre a liberação de hormônios e enzimas que darão início ao crescimento e desenvolvimento do grão \cite{LEWIS2012}. O teor de água no grão cresce rapidamente após o início dessa primeira etapa, que ocorre em duas fases distintas: uma inicial, em que o embrião e o escutelo absorvem água rapidamente, e uma segunda fase, mais lenta, em que o endosperma (\autoref{fig:graodecevada}) é gradualmente hidratado. Esse processo é crucial para a ativação de enzimas como a amilase, a ribonuclease e a fosfatase, que desempenham papéis essenciais na modificação do grão \cite{REYNOLDS1966}. 

Além disso, é essencial que o oxigênio seja fornecido ao mesmo tempo em que o dióxido de carbono é eliminado dos grãos. O aumento da umidade intensifica o metabolismo do grão, elevando sua taxa respiratória, ou seja, mais oxigênio é requerido \cite{KUNZE1996}. Por isso, a presença de tanques de maceração que bombeiam ar através dos grãos submersos é frequente nas produções de larga escala \cite{CENCI2021}. Mas, muitos estudos ainda investigam os impactos de se trabalhar com ciclos de períodos secos e submersos, especialmente, quando se busca viabilizar novas variedades de grãos para a malteação \cite{MAYER2014,TURNER2019}.

\begin{figure}[ht]
    \centering
    \caption{Estrutura do grão de cevada.}
    \label{fig:graodecevada}
    \includegraphics[width=0.9\textwidth]{graodecevada.png}

    {\centering\footnotesize Fonte: Autoria própria. Adaptado de \citeonline{LEWIS2012}.\par}
\end{figure}



\subsection{Germinação}

Ao fim da maceração, que pode durar até 72 horas, o teor de umidade dos grãos chega a aproximadamente 45\%, e as radículas tornam-se visíveis, indicando o momento adequado para o início da germinação. Essa segunda etapa é caracterizada por um elevado nível metabólico nos grãos, com a ocorrência de transformações bioquímicas essenciais para o processo de malteação \cite{MALLETT2022}. Do ponto de vista do malteador, a duração da germinação deve ser cuidadosamente controlada, pois um período insuficiente ou excessivo pode comprometer a qualidade do malte. Se a germinação for muito curta, as transformações necessárias nos grãos não ocorrerão de forma adequada. Por exemplo, as enzimas não conseguirão degradar completamente as paredes celulares proteicas que envolvem o amido, tornando-o menos disponível para as etapas subsequentes do processo cervejeiro \cite{FOX2009}. Por outro lado, uma germinação prolongada pode levar ao esgotamento excessivo dos nutrientes do grão, devido ao crescimento da planta, e, por consequência, afetar negativamente o produto final \cite{LEWIS2012}.

Nesta etapa, a ativação de enzimas hidrolíticas tem papel essencial na modificação do grão. As $\beta$-glucanases promovem a degradação dos $\beta$-glucanos (\autoref{fig:betaglucanos}) presentes na parede celular do endosperma, substâncias estas que são indesejadas no processo cervejeiro em altas concentrações \cite{LEWIS2012}. Essas enzimas hidrolisam as ligações $\beta$-1,3 e $\beta$-1,4 dos glucanos de cadeia mista, comprometendo a integridade da parede celular e facilitando a liberação de proteínas presentes nas células da camada de aleurona (\autoref{fig:graodecevada}) \cite{bobade2022betaglucans}. Paralelamente, proteases hidrolisam a matriz proteica que envolve os grânulos de amido, liberando pequenos peptídeos e aminoácidos \cite{FOX2009,GUPTA2010}. 

Já a conversão do amido (\autoref{fig:amido}) é mediada por enzimas amilolíticas, sendo a $\alpha$-amilase responsável pela quebra aleatória das ligações $\alpha$-1,4-glicosídicas e a $\beta$-amilase pela liberação de maltose, um dos principais açúcares fermentáveis do processo cervejeiro \cite{GUPTA2010,MALLETT2022}. Em essência, o grão cru entra no processo com compostos de alto peso molecular e, ao fim da malteação, gera como produto um malte com compostos de baixo peso molecular e boa concentração de enzimas \cite{KUNZE1996}. Essa transformação, que ocorre principalmente na germinação, é o que torna o malte indispensável para o processo cervejeiro \cite{CENCI2021}.

\begin{figure}[ht]
    \centering
    \caption{Esquema de uma molécula de $\beta$-glucano.}
    \label{fig:betaglucanos}
    \includegraphics[width=0.9\textwidth]{betaglucanos.jpg}

    {\centering\footnotesize Fonte: \citeonline{rop2009betaglucans}.\par}
\end{figure}

\begin{figure}[ht]
    \centering
    \caption{Estrutura química do amido representando as unidades de amilose (\textit{amylose}) e amilopectina (\textit{amylopectin}).}
    \label{fig:amido}
    \includegraphics[width=0.9\textwidth]{amido.png}

    {\centering\footnotesize Fonte: \citeonline{visakh2012starch}.\par}
\end{figure}

\begin{table}[ht]
    \caption{Comparativo entre as enzimas $\alpha$-amilase e $\beta$-amilase.}
    \label{tab:amilases}
    \centering
    \begin{tabular}{ccc}
        \hline
        \bfseries Característica & \bfseries $\alpha$-amilase & \bfseries $\beta$-amilase \\
        \hline
        Posição de ataque & Interna na cadeia & Extremos não redutores \\
        Produto principal & Maltose, glicose e dextrinas & Maltose \\
        Ligação hidrolisada & $\alpha$-1,4 & $\alpha$-1,4 \\
        Temperatura ideal & 70 -- 75 $^{\circ}$C & 60 -- 65 $^{\circ}$C \\
        Termoestabilidade & Alta & Moderada \\
        Papel no processo & Liquefação do amido & Produção de açúcares fermentáveis \\
        \hline
    \end{tabular}

    {\centering\footnotesize Fonte: Adaptado de \citeonline{GUPTA2010}, \citeonline{LEWIS2012}, \citeonline{MALLETT2022}.\par}
\end{table}



\subsection{Secagem}

Quando a germinação atinge o tempo otimizado para promover as melhores transformações, inicia-se a etapa de secagem. Nessa fase, de acordo com \apudonline{GRIFFITHS1992}{WOFFENDEN2002}, os grãos são desidratados por até 30 horas, o que resulta em um malte base de fácil manuseio e adequado para armazenamento. Além da redução da umidade para aumento da estabilidade do produto final, a secagem promove o desenvolvimento de aromas desejados e de coloração no malte \cite{BAMFORTH2003}. Apesar disso, se a intenção é produzir um malte base, a temperatura não pode ser excessiva, com a finalidade de preservar as enzimas, como as amilases, no produto final \cite{LEWIS2012}. Outra questão benéfica é que a secagem elimina boa parte dos microorganismos que cresceram de forma indesejada durante os processos anteriores \cite{DOUGLAS1988, PETTERS1988}. 



\section{Variáveis na malteação}

\subsection{Temperatura}

O controle da temperatura durante a germinação é essencial para minimizar perdas causadas pelo crescimento excessivo das radículas e do embrião da planta, evitando o consumo desnecessário dos estoques de amido do grão \cite{PITZ1990, MALLETT2022}. Além disso, a temperatura influencia diretamente a atividade enzimática e a degradação de componentes estruturais do grão. Um estudo conduzido por \citeonline{BAXTER1980} avaliou os efeitos da maceração em uma temperatura superior à faixa usual de 12-16 $^{\circ}$C, chegando a 30 $^{\circ}$C. Os resultados indicaram que temperaturas elevadas comprometem a atividade enzimática e reduzem a eficiência da degradação de $\beta$-glucanos e proteínas, afetando negativamente a qualidade do malte.

Outro fator crítico relacionado à temperatura é o crescimento microbiológico. Segundo \citeonline{TANGNI2002}, temperaturas mais altas na malteação favorecem a proliferação de \textit{Aspergillus clavatus}, um fungo produtor de micotoxinas. A contaminação microbiológica ocorre predominantemente durante a germinação, mas também pode ser observada ao final da maceração \cite{PETTERS1988}. Dessa forma, a manutenção de temperaturas controladas entre 12 e 22 $^{\circ}$C durante as etapas de maceração e germinação é fundamental não apenas para garantir a qualidade do malte, mas também para assegurar a segurança sanitária do processo \cite{TANGNI2002}.

\subsubsection{Temperatura na secagem}

Na etapa de secagem, o controle da temperatura desempenha um papel crucial em dois aspectos principais: garantir que o malte atinja a umidade adequada para armazenamento e determinar o tipo de malte produzido \cite{KUNZE1996}. A secagem ocorre, geralmente, em múltiplas fases, seguindo uma rampa de temperatura. Os valores típicos variam na faixa de 50 a 110 $^{\circ}$C, dependendo do perfil desejado para o malte final \cite{LEWIS2012}.

De acordo com \citeonline{SKENDI2018}, a temperatura de secagem influencia diretamente a composição de açúcares fermentáveis no malte e a coloração do mosto produzido. No estudo, a secagem a 80 $^{\circ}$C resultou em um mosto com maior teor de açúcares fermentáveis do que a secagem a temperaturas superiores, como 90 $^{\circ}$C. Além disso, foi observado um escurecimento do mosto com o aumento da temperatura de secagem, um fator determinante na definição das características finais do malte. Essa relação entre temperatura e cor ocorre devido à intensificação das reações de Maillard e à degradação térmica de compostos presentes no malte \cite{KUNZE1996}.

\subsection{Aeração}

É de suma importância que os grãos estejam bem aerados, uma vez que o processo envolve um organismo vivo que depende da respiração para seu desenvolvimento \cite{MALLETT2022}. De acordo com \citeonline{WILHELMSOM2006}, no início da maceração ocorre uma deficiência de oxigênio devido à submersão dos grãos, mas esse fator não compromete a qualidade do malte. No entanto, à medida que a germinação avança, a disponibilidade de O$_2$ torna-se mais crítica, pois a respiração dos grãos gera acúmulo de dióxido de carbono, o que pode inibir a germinação. Esse efeito pode ser intensificado pelo emaranhamento das radículas, que dificulta a circulação de ar e reduz ainda mais a disponibilidade de oxigênio. Para mitigar esse problema, sistemas revolvedores são frequentemente empregados para movimentar os grãos e garantir uma aeração adequada ao longo do processo \cite{CENCI2021}. 

\subsection{Tempo}
A duração de cada etapa da malteação é um fator crítico para a qualidade do malte, influenciada pela variedade do grão e pelas condições do processo (umidade, temperatura e aeração). A otimização desse parâmetro é essencial para adaptar novas variedades às demandas industriais. Como demonstrado por \citeonline{FARNAZEH2017}, o tempo de germinação (3 a 7 dias) afeta diretamente as propriedades do malte: períodos mais longos (7 dias) elevam a atividade de $\beta$-glucanase e $\alpha$-amilase, reduzindo o teor de amido e $\beta$-glucano devido ao consumo enzimático, além de aumentar as perdas por crescimento. Assim, a definição do tempo ideal deve equilibrar modificação enzimática e eficiência do processo, considerando o perfil desejado no malte. Por exemplo, germinação prolongada é indicada para cervejas de alta fermentabilidade, enquanto períodos mais curtos (3-5 dias) preservam polissacarídeos, sendo ideais para estilos encorpados.

Na etapa de maceração, o tempo necessário para a hidratação dos grãos é um fator determinante para a ativação enzimática e o desenvolvimento adequado do malte. \citeonline{MONTANUCI2017} demonstraram que períodos mais longos de maceração favorecem a absorção de água, impactando diretamente a degradação de $\beta$-glucanos e o desenvolvimento das enzimas $\alpha$- e $\beta$-amilase. No entanto, o tempo ideal depende da temperatura empregada: enquanto macerações a 10 $^{\circ}$C por 24 horas resultam em maior teor de açúcares no malte, temperaturas mais elevadas (20 $^{\circ}$C por 12 horas) aceleram a hidratação, porém podem comprometer a viabilidade dos grãos e aumentar a degradação de componentes essenciais. Além disso, tempos excessivos de maceração podem favorecer o crescimento microbiológico indesejado, exigindo um controle rigoroso para evitar contaminações e perdas na qualidade do malte \cite{LUARASI2016}. Dessa forma, a definição do tempo de maceração deve equilibrar a eficiência da hidratação com a preservação da integridade dos grãos, garantindo um substrato adequado para as etapas subsequentes da malteação.

O tempo de secagem deve ser ajustado em conjunto com a temperatura para garantir uma remoção eficiente da umidade, reduzindo-a para aproximadamente 4\%, sem comprometer a qualidade enzimática e sensorial do malte \cite{LEWIS2012}.



\section{Automação e controle de processos}

A automação desempenha um papel essencial na indústria no geral, proporcionando maior qualidade no produto final, otimização da produção e aumento da segurança operacional \cite{SEBORG2016}. Para garantir um controle eficaz dos processos industriais, são utilizados sistemas instrumentados, compostos por três elementos fundamentais: sensores, processadores de sinais e interfaces de visualização \cite{BOLTON2021}. 

Os sensores são responsáveis pela coleta de informações sobre variáveis do processo, como temperatura, pressão e vazão, permitindo o monitoramento contínuo das condições operacionais. Os processadores de sinais, por sua vez, realizam o tratamento e a interpretação dos dados captados, aplicando algoritmos de controle para ajustar automaticamente os parâmetros do sistema conforme necessário. Já as interfaces de visualização possibilitam que operadores e engenheiros acompanhem o comportamento do processo em tempo real, viabilizando a tomada de decisões informadas e a rápida identificação de falhas \cite{BOLTON2021}. Além disso, sistemas automatizados contribuem significativamente para a reprodutibilidade dos processos industriais, reduzindo variações indesejadas e garantindo conformidade com padrões regulatórios \cite{SEBORG2016}.

\subsection{Controlador ON-OFF}
O controlador ON-OFF é um dos métodos mais simples de controle de processos, operando com apenas dois estados: ligado (ON) e desligado (OFF). Esse tipo de controle é amplamente utilizado quando precisão extrema não é necessária e o sistema pode tolerar pequenas oscilações na variável controlada \cite{BOLTON2021}. Seu funcionamento é baseado em um ponto de ajuste (\textit{setpoint}): quando a variável de processo ultrapassa esse valor, o atuador é ativado ou desativado, sem intermediários. Embora seja uma solução de fácil implementação e baixo custo, pode gerar oscilações constantes em torno do \textit{setpoint}, tornando-se inadequado para processos que exigem estabilidade mais refinada \cite{SEBORG2016}.

\subsection{Interfaces Gráficas}
A Interface Gráfica do Usuário, ou HMI (\textit{Human-Machine Interface} - Interface Homem Máquina), é um elemento essencial na automação, pois permite a interação intuitiva entre operadores e sistemas de controle. Segundo \citeonline{BOLTON2021}, esse tipo de interface apresenta informações de processo de forma visual e interativa, utilizando janelas, ícones, menus e dispositivos apontadores. Em ambientes industriais, as telas de supervisão geralmente empregam displays miméticos, que representam esquematicamente as principais partes de uma planta, exibindo valores atualizados de variáveis controladas, gráficos de tendências e alarmes em tempo real. Além disso, as interfaces gráficas possibilitam que os operadores definam valores de \textit{setpoint} e ajustem parâmetros do processo diretamente pela tela, eliminando a necessidade de comandos textuais complexos.



\section{Microcontroladores da Espressif (China)}

De acordo com \citeonline{KOLBAN2017}, os circuitos integrados baseados no microcontrolador ESP32, da Espressif (China), são amplamente utilizados devido ao seu baixo custo e à capacidade de executar aplicações autônomas, tornando-se uma escolha popular para projetos de automação e controle de processos. Além disso, sua versatilidade permite a integração com diversos sensores e dispositivos periféricos, tornando-o adequado para uma ampla gama de aplicações em automação e sistemas embarcados.

Graças à sua conectividade com a internet, as placas ESP32 estão fortemente associadas ao conceito de IoT. O IoT tem papel fundamental na otimização de processos industriais, permitindo a coleta de dados em larga escala (\textit{big data}) e a detecção de falhas, o que contribui para a redução de custos operacionais \cite{ferencz2020rapid}. Essa tecnologia é composta por uma variedade de dispositivos, como sensores, atuadores e controladores, que podem se comunicar por meio de diferentes protocolos, incluindo Bluetooth, Ethernet e Wi-Fi \cite{shinde2017industrial}.

Um exemplo representativo dessa linha é o \textbf{ESP32-C3}, também desenvolvido pela Espressif Systems, que combina conectividade sem fio com processamento eficiente em um encapsulamento compacto. Esse modelo integra um núcleo RISC-V de 32 bits operando a até 160~MHz, além de periféricos como UART, SPI, I²C, temporizadores, controladores PWM e ADCs de 12 bits com até seis canais (\autoref{fig:esp32c3}). A memória interna é composta por 400~kB de SRAM, 384~kB de ROM e 8~kB adicionais dedicados à operação em modo de baixo consumo. A operação típica do chip ocorre com alimentação de 3.3~V, com suporte a múltiplos modos de economia de energia, incluindo o \textit{deep-sleep}, no qual o consumo é reduzido à ordem de microamperes \cite{espressif_esp32c3_2025}.

\begin{figure}[ht]
    \centering
    \caption{Esquema de uma placa com chip ESP32-C3FH4 (WeActStudio, China) }
    \label{fig:esp32c3}
    \includegraphics[width=0.9\textwidth]{diagrama-esp32-c3.jpg}

    {\centering\footnotesize Fonte: Devicemart.\par}
\end{figure}

A flexibilidade dos pinos de entrada e saída (\autoref{fig:esp32c3}) programáveis, somada à capacidade de comunicação via Wi-Fi e Bluetooth Low Energy, tornam esses microcontroladores uma alternativa robusta e econômica para aplicações embarcadas em automação, monitoramento remoto e sistemas de controle \cite{espressif_esp32c3_2025}.


\subsection{MicroPython}

O MicroPython é um interpretador da linguagem Python 3 desenvolvido especificamente para microcontroladores \cite{PLAUSKA2022}. A linguagem Python destaca-se por sua sintaxe simplificada e ampla adoção na comunidade tecnológica, sendo especialmente útil para programadores iniciantes \cite{TOLLERVEY2017}.

Estudos conduzidos por \citeonline{PLAUSKA2022} avaliaram o desempenho das principais linguagens utilizadas na programação do ESP32, incluindo o MicroPython. Os resultados indicaram que, embora sua eficiência em termos de desempenho seja inferior à de linguagens compiladas, como C e C++, o MicroPython continua sendo uma alternativa viável para aplicações onde o controle direto e preciso do hardware não é essencial. Sua principal vantagem reside na facilidade de prototipação e desenvolvimento rápido, tornando-o ideal para projetos com foco em lógica de alto nível ou ensino \cite{tanganelli2019rapid}.



\section{Android}

O Android é um dos principais sistemas operacionais para dispositivos móveis, sendo um projeto de código aberto baseado no kernel Linux e liderado pelo Google \cite{ABLESON2011}. Inicialmente, os aplicativos desenvolvidos para essa plataforma eram escritos predominantemente em Java \cite{ABLESON2011}. No entanto, em 2017, o Google anunciou o suporte oficial ao Kotlin como linguagem de programação integrada ao \textit{Android SDK}, permitindo que os desenvolvedores utilizem a linguagem nativamente no ambiente de desenvolvimento padrão da plataforma \cite{SILLS2023}.

A integração entre sistemas IoT e dispositivos móveis baseados em Android representa um avanço significativo para aplicações de monitoramento remoto. Diferentes estudos demonstram a viabilidade do uso de smartphones como interface de visualização e controle em arquiteturas distribuídas. 

\citeonline{mohanasundaram2024water} apresentaram um sistema para monitoramento da qualidade da água em tempo real, com sensores de pH, turbidez, temperatura e nível, cujos dados eram transmitidos para uma aplicação Android, permitindo o acompanhamento contínuo e remoto dos parâmetros analisados. De forma semelhante, \citeonline{do2021internet} desenvolveram um sistema para monitoramento do nível de água em reservatórios, empregando sensores ultrassônicos e um microcontrolador ESP8266, com os dados sendo exibidos em gráficos em um aplicativo Android. O uso do dispositivo móvel como interface contribuiu para facilitar a visualização remota e a interação com o sistema, demonstrando sua aplicabilidade em soluções IoT voltadas ao monitoramento contínuo.

Outro caso representativo foi descrito por \citeonline{dhingra2019internet}, que implementaram um sistema de monitoramento da qualidade do ar com sensores integrados a uma plataforma em nuvem, permitindo a consulta do índice de qualidade do ar em tempo real por meio de um aplicativo Android. O sistema utiliza geolocalização para gerar rotas personalizadas com base na poluição atmosférica, incluindo mapas interativos com alertas visuais. Tais abordagens reforçam o papel do Android como plataforma acessível e eficiente para aplicações embarcadas conectadas ao IoT.

\subsection{Kotlin}

O Kotlin se destaca como uma linguagem moderna, concisa, segura e pragmática, oferecendo total interoperabilidade com código Java, o que facilita a migração e a integração de projetos legados \cite{JEMEROV2017}. Atualmente, é amplamente adotado como a principal linguagem para o desenvolvimento de aplicativos móveis na plataforma Android, proporcionando maior produtividade e reduzindo a incidência de erros comuns durante a programação. Essa adoção é fortalecida pela integração nativa do Kotlin no \textit{Android Studio}, ambiente de desenvolvimento oficial mantido pelo Google, que fornece ferramentas específicas para facilitar a escrita, o teste e a depuração de aplicativos Android \cite{ANDROIDSTUDIO}.

\subsection{Bluetooth Low Energy}

A plataforma Android oferece suporte a diferentes protocolos de comunicação via Bluetooth, incluindo o Bluetooth Low Energy (BLE), uma tecnologia especialmente adequada para conexões eficientes com dispositivos de baixo consumo energético, como o ESP32 \cite{android_bluetooth_overview}.

Desenvolvido como uma evolução do Bluetooth clássico, o BLE foi projetado para reduzir significativamente o consumo de energia sem comprometer a conectividade \cite{heydon2012bluetooth}. Essa característica o torna ideal para aplicações que demandam comunicação contínua com eficiência energética, como sistemas de automação e monitoramento remoto. Nesse contexto, a integração entre dispositivos Android (como smartphones) e microcontroladores (como o ESP32) via BLE viabiliza a troca de dados em tempo real, dispensando a necessidade de conexões Wi-Fi ou cabos.

A comunicação BLE baseia-se em uma arquitetura assimétrica, envolvendo dois tipos de dispositivos: o \textit{central} (por exemplo, um smartphone Android) e o \textit{periférico} (como o ESP32). Como ilustrado na \autoref{fig:ConexaoBLE}, o processo inicia-se com o envio de pacotes de anúncio (\textit{advertising packets}) pelo periférico, que sinaliza sua disponibilidade. O dispositivo central, ao realizar uma varredura (\textit{scanning}), detecta esses pacotes e pode solicitar uma conexão. Uma vez estabelecido o pareamento, o central identifica os serviços disponíveis no periférico e inicia a troca estruturada de dados. Além disso, esse modelo permite que o periférico otimize o consumo de energia, mantendo seu rádio desligado durante períodos de inatividade, enquanto o central assume as operações mais complexas \cite{bluetoothLEprimer2024}.

\begin{figure}[ht]
    \centering
    \caption{Ilustração da conexão BLE de um dispositivo periférico (ESP32) com um celular (Android)}
    \label{fig:ConexaoBLE}
    \includegraphics[width=0.8\textwidth]{ConexaoBLE.drawio.png}

    {\centering\footnotesize Fonte: Autoria própria.\par}
\end{figure}

A estrutura do BLE é organizada em camadas, com destaque para o perfil GATT (\textit{Generic Attribute Profile}), responsável pela organização e transmissão dos dados após a conexão. No GATT, as informações são agrupadas em serviços, que contêm características individuais identificadas por UUIDs (\textit{Universally Unique Identifiers}). Por exemplo, um serviço dedicado a medições de temperatura pode incluir uma característica específica para o valor da leitura. A \autoref{fig:gattestrutura} detalha essa hierarquia, mostrando como os perfis são compostos por serviços e características, cada uma com propriedades definidas (como leitura, escrita ou notificação). Essa padronização permite que dispositivos centrais, como smartphones Android, acessem dados de forma consistente, independentemente do periférico utilizado \cite{bleGATT}.

\begin{figure}[ht]
    \centering
    \caption{Hierarquia GATT}
    \label{fig:gattestrutura}
    \includegraphics[width=0.6\textwidth]{BLE.png}

    {\centering\footnotesize Fonte: Bluetooth.\par}
\end{figure}