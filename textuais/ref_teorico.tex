\chapter[Referencial Teórico]{Referencial Teórico}

\section{O Processo de Malteação}

A malteação é um dos processos fundamentais na indústria cervejeira, responsável pela conversão do grão cru em um ingrediente essencial para a produção de cerveja. Esse processo ocorre em três etapas principais: maceração, germinação e secagem. Essas etapas são fundamentais para a produção de malte base: um malte que possui uma boa quantidade de enzimas e estoques de amido \cite{BRIGGS2004,CENCI2021}. Além disso, pode-se considerar uma quarta etapa no processo: a torrefação, que ocorre após a secagem e produz um malte especial no lugar de um malte base. Maltes especiais são conhecidos por serem formados com bastante influência das reações de Maillard \cite{COGHE2004}. A complexidade desse processo se dá no fato da manipulação de um organismo vivo, o grão, que exige controle rigoroso de suas condições de crescimento para se tornar um produto viável para comercialização \cite{MALLETT2022}. Além disso, em condições inadequadas de malteação, podem ocorrer ataques microbiológicos de diversas espécies, incluindo Fusarium sp., Penicillium e Aspergillus genera \cite{LUARASI2016}.

\subsection{Maceração}

Durante a maceração, com a submersão do grão em água, ocorre a liberação de hormônios e enzimas que darão início ao crescimento e desenvolvimento do grão \cite{LEWIS2012}. O teor de água no grão cresce rapidamente após o início dessa primeira etapa, que ocorre em duas fases distintas: uma inicial, em que o embrião e o escutelo absorvem água rapidamente, e uma segunda fase, mais lenta, em que o endosperma é gradualmente hidratado. Esse processo é crucial para a ativação de enzimas como a amilase, a ribonuclease e a fosfatase, que desempenham papeis essenciais na modificação do grão \cite{REYNOLDS1966}. Também é crucial que o oxigênio seja fornecido ao mesmo tempo em que o dióxido de carbono é eliminado dos grãos. O aumento da umidade intensifica o metabolismo do grão, elevando sua taxa respiratória, ou seja, mais oxigênio é requerido \cite{KUNZE1996}. Por isso, a presença de tanques de maceração que bombeiam ar através dos grãos submersos é frequente nas produções de larga escala \cite{CENCI2021}. Mas, muitos estudos ainda investigam os impactos de se trabalhar com ciclos de períodos secos e submersos, especialmente, quando se busca viabilizar novas variedades de grãos para a malteação \cite{MAYER2014,TURNER2019}.

\subsection{Germinação}

Ao fim da maceração, que pode durar até 72 horas, o teor de umidade dos grãos chega a aproximadamente 45\%, e as radículas tornam-se visíveis, indicando o momento adequado para o início da germinação. Essa segunda etapa é caracterizada por um elevado nível metabólico nos grãos, com a ocorrência de transformações bioquímicas essenciais para o processo de malteação \cite{MALLETT2022}. Do ponto de vista do malteador, a duração da germinação deve ser cuidadosamente controlada, pois um período insuficiente ou excessivo pode comprometer a qualidade do malte. Se a germinação for muito curta, as transformações necessárias nos grãos não ocorrerão de forma adequada. Por exemplo, as enzimas não conseguirão degradar completamente as paredes celulares proteicas que envolvem o amido, tornando-o menos disponível para as etapas subsequentes do processo cervejeiro \cite{FOX2009}. Por outro lado, uma germinação prolongada pode levar ao esgotamento excessivo dos nutrientes do grão visando o crescimento da planta, afetando negativamente o produto final \cite{LEWIS2012}.

Durante a germinação, a ativação de enzimas hidrolíticas desempenha um papel essencial na modificação do grão.As $\beta$-glucanases promovem a degradação dos $\beta$-glucanos presentes na parede celular do endosperma, substâncias estas que são indesejadas no processo cervejeiro em altas concentrações \cite{LEWIS2012}. Paralelamente, proteases hidrolisam a matriz proteica que envolve os grânulos de amido, liberando pequenos peptídeos e aminoácidos \cite{FOX2009,GUPTA2010}. Já a conversão do amido é mediada por enzimas amilolíticas,sendo a $\alpha$-amilase responsável pela quebra aleatória das ligações $\alpha$-1,4-glicosídicas e a $\beta$-amilase pela liberação de maltose, um dos principais açúcares fermentáveis do processo cervejeiro \cite{GUPTA2010,MALLETT2022}. Em essência, o grão cru entra no processo com compostos de alto peso molecular e, ao fim da malteação, gera como produto um malte com compostos de baixo peso molecular e boa concentração de enzimas \cite{KUNZE1996}. Essa transformação, que ocorre principalmente na germinação, é o que torna o malte indispensável para o processo cervejeiro \cite{CENCI2021}.

\subsection{Secagem}

Quando a germinação atinge o tempo otimizado para promover as melhores transformações, inicia-se a etapa de secagem. Nessa fase, de acordo com \apudonline{GRIFFITHS1992}{WOFFENDEN2002}, os grãos são desidratados por até 30 horas, o que resulta em um malte base de fácil manuseio e adequado para armazenamento. Além da redução da umidade para aumento da estabilidade do produto final, a secagem promove o desenvolvimento de aromas desejados e de coloração no malte \cite{BAMFORTH2003}. Apesar disso, se a intenção é produzir um malte base, a temperatura não pode ser excessiva, com o fim de preservar as enzimas, como as amilases, no produto final \cite{LEWIS2012}.Outra questão benéfica é que a secagem elimina boa parte dos microorganismos que cresceram de forma indesejada durante os processos anteriores \cite{DOUGLAS1988, PETTERS1988}. 


\section{Importância do controle de variáveis na malteação}

\subsection{Controle de temperatura}

O controle da temperatura durante a germinação é essencial para minimizar perdas causadas pelo crescimento excessivo das radículas e do embrião da planta, evitando o consumo desnecessário dos estoques de amido do grão \cite{PITZ1990, MALLETT2022}. Além disso, a temperatura influencia diretamente a atividade enzimática e a degradação de componentes estruturais do grão. Um estudo conduzido por \citeonline{BAXTER1980} avaliou os efeitos da maceração em uma temperatura superior à faixa usual de 12-16 $^{\circ}$C, chegando a 30 $^{\circ}$C. Os resultados indicaram que temperaturas elevadas comprometem a atividade enzimática e reduzem a eficiência da degradação de $\beta$-glucanos e proteínas, afetando negativamente a qualidade do malte.

Outro fator crítico relacionado à temperatura é o crescimento microbiológico. Segundo \citeonline{TANGNI2002}, temperaturas mais altas na malteação favorecem a proliferação de \textit{Aspergillus clavatus}, um fungo produtor de micotoxinas. A contaminação microbiológica ocorre predominantemente durante a germinação, mas também pode ser observada ao final da maceração \cite{PETTERS1988}. Dessa forma, a manutenção de temperaturas controladas entre 12 e 22 $^{\circ}$C durante as etapas de maceração e germinação é fundamental não apenas para garantir a qualidade do malte, mas também para assegurar a segurança sanitária do processo \cite{TANGNI2002}.

\subsubsection{Controle de temperatura na secagem}

Na etapa de secagem, o controle da temperatura desempenha um papel crucial em dois aspectos principais: garantir que o malte atinja a umidade adequada para armazenamento e determinar o tipo de malte produzido \cite{KUNZE1996}. A secagem ocorre, geralmente, em múltiplas fases, seguindo uma rampa de temperatura. Os valores típicos variam na faixa de 50 a 110 $^{\circ}$C, dependendo do perfil desejado para o malte final \cite{LEWIS2012}.

De acordo com \citeonline{SKENDI2018}, a temperatura de secagem influencia diretamente a composição de açúcares fermentáveis no malte e a coloração do mosto produzido. No estudo, a secagem a 80 $^{\circ}$C resultou em um mosto com maior teor de açúcares fermentáveis do que a secagem a temperaturas superiores, como 90 $^{\circ}$C. Além disso, foi observado um escurecimento do mosto com o aumento da temperatura de secagem, um fator determinante na definição das características finais do malte. Essa relação entre temperatura e cor ocorre devido à intensificação das reações de Maillard e à degradação térmica de compostos presentes no malte \cite{KUNZE1996}.

\subsection{Temp}
% Como temperatura, umidade e CO₂ afetam a qualidade do malte