%% Macros de dados do documento
\autor{Nícolas Dezan dos Santos}
\titulo{AUTOMAÇÃO DE UM MALTEADOR LABORATORIAL: DESENVOLVIMENTO DE FIRMWARE E APLICATIVO PARA CONTROLE DE PROCESSOS}
\instituicao{Instituto Federal do Espírito Santo}
\newcommand{\campus}{Campus Vila Velha}
\newcommand{\imprimircampus}{\campus}
\curso{Curso de Química Industrial}
\local{Vila Velha-ES}
\data{2025}


\tipotrabalho{Trabalho de Graduação}
\preambulo{Trabalho de Conclusão de Curso apresentado à Coordenadoria do Curso de Química Industrial do
Instituto Federal de Educação, Ciência e Tecnologia do Espírito Santo, como requisito parcial para a obtenção do título de Químico Industrial.}

\orientador[Orientador][Prof. Dr.]{Ernesto}[Corrêa Ferreira]{Instituto Federal do Espírito Santo}
\coorientador[][]{}[]{}

\examinadori{Profa. Dra. Fulana de Tal}{Instituto Federal do Espírito Santo}{Examinadora} % DEPOIS
\examinadorii{Prof. Dr. Cicrano de Tal}{Instituto Federal do Espírito Santo}{Examinador} % DEPOIS

\approvaldate{32}{Dezembro}{2999} % DEPOIS

% Palavras-chaves
\palavraschave{Automação}[Malteação][Firmware][Android][Controle de Processos]
\keywords{Automation}[Malting][Firmware][Android][Process Control]

% Ficha catalográfica
\fichabiblioteca{
    Dados Internacionais de Catalogação-na-Publicação (CIP) \\
    (Biblioteca Nilo Peçanha do Instituto Federal do Espírito Santo) % Vish
}
\fichaautor{Elaborada por XXXXXXXXXXXXXXXXXXXXX – CRB-X/ES - XXX}
\fichatags{
    1. Redes neurais (Computação).  2. Expressão facial – Processamento de dados. 3. Sistemas de reconhecimento de padrões. 4. Processamento de imagens – Técnicas digitais. 5. Percepção de padrões. 6. Engenharia Elétrica. I. XXXXXXXX, XXXXXXXXXX XXXXXXXXXX.  II. Instituto Federal do Espírito Santo. III. Título.
}

\cutter{X999y}
\cdd{000.00}
\selectlanguage{brazil}

% Arial (para usar times-new-roman, comente as três linhas abaixo)
\usepackage{helvet}
\renewcommand{\familydefault}{\sfdefault}
\renewcommand{\ABNTEXchapterfont}{\bfseries}
% Comandos para revisar o texto
\usepackage{easyReview} 


%% Elementos opcionais
\editardedicatoria{\input{pre_textuais/dedicatoria}}
\editaragradecimentos{\input{pre_textuais/agradecimentos}}
\editarepigrafe{\input{pre_textuais/epigrafe}}


%% Listas [Siglas e Simbolos]
\editarlistasiglas{\input{pre_textuais/siglas}}
\editarlistasimbolos{\input{pre_textuais/simbolos}}